
\documentclass[12pt]{article}
\usepackage[brazil]{babel}
\usepackage[utf8]{inputenc}
\usepackage{geometry}
\geometry{a4paper, margin=2.5cm}
\usepackage{enumitem}
\usepackage{titlesec}
\usepackage{fancyhdr}
\usepackage{listings}
\usepackage{color}
\usepackage{graphicx}

\title{Documentação Técnica - Projeto MVC}
\author{Equipe Last Seven}
\date{\today}

\pagestyle{fancy}
\fancyhf{}
\rhead{Last Seven}
\lhead{Documentação Técnica}
\rfoot{\thepage}

\titleformat{\section}{\normalfont\Large\bfseries}{\thesection}{1em}{}
\titleformat{\subsection}{\normalfont\large\bfseries}{\thesubsection}{1em}{}

\begin{document}

\maketitle

\section{Introdução}
\begin{itemize}
  \item \textbf{Nome do Projeto:} Last Seven Portfolio Website
  \item \textbf{Descrição:} Website desenvolvido pela equipe Last Seven com o objetivo de apresentar um portfólio profissional individual de cada um dos 7 integrantes da empresa.
  \item \textbf{Objectivo do Sistema:} Permitir que clientes conheçam o trabalho da equipe Last Seven e possam solicitar serviços através de um formulário de contacto.
  \item \textbf{Tecnologias Utilizadas:} HTML, CSS, JavaScript, PHP, MySQL.
  \item \textbf{Arquitectura Utilizada:} MVC (Model-View-Controller)
\end{itemize}

\section{Requisitos do Sistema}

\subsection{Requisitos Funcionais}
\begin{itemize}
  \item RF01 – O sistema deve exibir o portfólio de cada integrante.
  \item RF02 – O sistema deve permitir o envio de mensagens via formulário de contacto.
  \item RF03 – O sistema deve armazenar as mensagens dos clientes na base de dados.
\end{itemize}

\subsection{Requisitos Não-Funcionais}
\begin{itemize}
  \item RNF01 – O sistema deve estar disponível 24 horas por dia.
  \item RNF02 – O sistema deve ser responsivo.
\end{itemize}

\section{Estrutura de Pastas (Arquitectura do Projeto)}
\begin{lstlisting}[language=bash]
/raiz-do-projeto
│
├── /model/         ← Regras de negócio, conexão com a base de dados
├── /view/          ← Interfaces visuais (HTML/CSS)
├── /controller/    ← Lógica de controle e ponte entre view e model
├── /config/        ← Configurações do sistema e da base de dados
├── /public/        ← Recursos acessíveis ao utilizador (imagens, scripts, etc.)
└── index.php       ← Ponto de entrada da aplicação
\end{lstlisting}

\section{Descrição dos Componentes}

\subsection{Model}
Responsável por lidar com os dados, fazer conexões com a base de dados e aplicar regras de negócio.

Exemplo de classes model: \texttt{Usuario.php}, \texttt{Projeto.php}, \texttt{MensagemContato.php}

\subsection{View}
Camada de apresentação, exibe os dados ao utilizador e recebe interações.

Exemplo de ficheiros: \texttt{home.php}, \texttt{portfolio.php}, \texttt{formulario-contato.php}

\subsection{Controller}
Recebe as requisições da view, interage com o model e devolve uma resposta adequada.

Exemplo de controladores: \texttt{UsuarioController.php}, \texttt{ContatoController.php}

\section{Fluxo de Funcionamento (MVC)}
\begin{enumerate}
  \item O utilizador interage com a \textbf{View}.
  \item A \textbf{View} envia os dados ao \textbf{Controller}.
  \item O \textbf{Controller} processa a lógica e chama o \textbf{Model}, se necessário.
  \item O \textbf{Model} interage com a base de dados e retorna os dados ao \textbf{Controller}.
  \item O \textbf{Controller} atualiza a \textbf{View} com os resultados.
\end{enumerate}

\section{Base de Dados}
\textbf{Tabelas principais:}
\begin{itemize}
  \item \texttt{usuarios(id, nome, email, senha)}
  \item \texttt{portfolios(id, usuario_id, descricao, imagem)}
  \item \texttt{mensagens(id, nome, email, assunto, mensagem, data\_envio)}
\end{itemize}

\section{Casos de Uso}
(Adicionar diagramas ou descrições dos principais casos de uso do sistema.)

\section{Testes}
Testes manuais realizados para verificar funcionamento do formulário e exibição do portfólio.

\section{Conclusão}
O sistema apresenta um site funcional com área de portfólio individual e formulário de contacto operante via backend em PHP com banco de dados MySQL.

\section{Anexos}
Capturas de tela, códigos relevantes, link do repositório no GitHub.

\end{document}
